\usepackage{listings} % Required for insertion of code
\usepackage[usenames,dvipsnames]{color} % Required for custom colors
\usepackage{courier} % Required for the courier font


%--------------------------------------------------------------------------
%	CODE INCLUSION CONFIGURATION
%--------------------------------------------------------------------------
% This is the color used for comments
\definecolor{MyDarkGreen}{rgb}{0.0,0.4,0.0} 
\lstloadlanguages{Perl,Python} % Load Perl syntax for listings,
% for a list of other languages supported see:
%ftp://ftp.tex.ac.uk/tex-archive/macros/latex/contrib/listings/listings.pdf
\lstset{language=Perl, % Use Perl in this example
        frame=single, % Single frame around code
        basicstyle=\small\ttfamily, % Use small true type font
        keywordstyle=[1]\color{Blue}\bf, % Perl functions bold and blue
        keywordstyle=[2]\color{Purple}, % Perl function arguments purple
        % Custom functions underlined and blue
        keywordstyle=[3]\color{Blue}\underbar, 
        identifierstyle=, % Nothing special about identifiers
        % Comments small dark green courier font
        commentstyle=\usefont{T1}{pcr}{m}{sl}\color{MyDarkGreen}\small, 
        stringstyle=\color{Purple}, % Strings are purple
        showstringspaces=false, % Don't put marks in string spaces
        tabsize=5, % 5 spaces per tab
        %
        % Put standard Perl functions not included
        % in the default language here
        morekeywords={rand},
        %
        % Put Perl function parameters here
        morekeywords=[2]{on, off, interp},
        %
        % Put user defined functions here
        morekeywords=[3]{test},
       	%
        % Line continuation (...) like blue comment
        morecomment=[l][\color{Blue}]{...}, 
        numbers=left, % Line numbers on left
        firstnumber=1, % Line numbers start with line 1
        numberstyle=\tiny\color{Blue}, % Line numbers are blue and small
        stepnumber=5 % Line numbers go in steps of 5
}


\lstset{language=Python, % Use Python in this example
        frame=single, % Single frame around code
        basicstyle=\small\ttfamily, % Use small true type font
        keywordstyle=[1]\color{Blue}\bf, % Python functions bold and blue
        keywordstyle=[2]\color{Purple}, % Python function arguments purple
        % Custom functions underlined and blue
        keywordstyle=[3]\color{Blue}\underbar, 
        identifierstyle=, % Nothing special about identifiers
        % Comments small dark green courier font
        commentstyle=\usefont{T1}{pcr}{m}{sl}\color{MyDarkGreen}\small, 
        stringstyle=\color{Purple}, % Strings are purple
        showstringspaces=false, % Don't put marks in string spaces
        tabsize=3, % 5 spaces per tab
        %
        % Put standard Python functions not included in the
        % default language here
        morekeywords={rand},
        %
        % Put Python function parameters here
        morekeywords=[2]{on, off, interp},
        %
        % Put user defined functions here
        morekeywords=[3]{test},
       	%
        % Line continuation (...) like blue comment
        morecomment=[l][\color{Blue}]{...}, 
        numbers=left, % Line numbers on left
        firstnumber=1, % Line numbers start with line 1
        numberstyle=\tiny\color{Blue}, % Line numbers are blue and small
        stepnumber=5 % Line numbers go in steps of 5
}


% Creates a new command to include a perl script, the first
% parameter is the filename of the script (without .pl), the
% second parameter is the caption
\newcommand{\perlscript}[2]{
\begin{itemize}
\item[]\lstinputlisting[caption=#2,label=#1]{#1.pl}
\end{itemize}
}

%New command for a python script
\newcommand{\pythonscript}[2]{
\begin{itemize}
\item[]\lstinputlisting[caption=#2,label=#1]{#1}
\end{itemize}
}

% Creates a new command to include a script, the first parameter is the filename of the script (without .m), the second parameter is the caption
\newcommand{\script}[5]{
\begin{itemize}
\item[]\lstinputlisting[caption=#2,label=#1,firstline=#3,lastline=#4,firstnumber=#5]{#1.py}
\end{itemize}
}




%%%% Use where you want the code inseted
\pythonscript{Problem4/Calculations}{Python Script for problem}



