%%%%%%%%%%%%%%%%%%%%%%%%%%%%%%%%%%%%%%%%%
% Programming/Coding Assignment
% LaTeX Template
%
% This template has been downloaded from:
% http://www.latextemplates.com
%
% Original author:
% Ted Pavlic (http://www.tedpavlic.com)
%
% Note:
% The \lipsum[#] commands throughout this template generate dummy text
% to fill the template out. These commands should all be removed when 
% writing assignment content.
%
% This template uses a Perl script as an example snippet of code,
% most other languages are also usable. Configure them in the
% "CODE INCLUSION CONFIGURATION" section.
%%%%%%%%%%%%%%%%%%%%%%%%%%%%%%%%%%%%%%%%%

%-------------------------------------------------------------------------
%	PACKAGES AND OTHER DOCUMENT CONFIGURATIONS
%-------------------------------------------------------------------------

\documentclass{article}

\usepackage{fancyhdr} % Required for custom headers
\usepackage{lastpage} % Required to determine the last page for the footer
\usepackage{extramarks} % Required for headers and footers
\usepackage[usenames,dvipsnames]{color} % Required for custom colors
\usepackage{graphicx} % Required to insert images
\usepackage{listings} % Required for insertion of code
\usepackage{courier} % Required for the courier font
\usepackage{hyperref}
\usepackage{amsmath}
\usepackage{mathrsfs}
% Margins
\topmargin=-0.45in
\evensidemargin=0in
\oddsidemargin=0in
\textwidth=6.5in
\textheight=9.0in
\headsep=0.25in

\linespread{1.1} % Line spacing

% Set up the header and footer
\pagestyle{fancy}
\lhead{\hmwkAuthorName} % Top left header
\chead{\hmwkClassShort\ (\hmwkClassInstructor)} % Top center head
%\rhead{\firstxmark} % Top right header
\rhead{\hmwkTitle}
\lfoot{\lastxmark} % Bottom left footer
\cfoot{} % Bottom center footer
% Bottom right footer
\rfoot{Page\ \thepage\ of\ \protect\pageref{LastPage}} 
\renewcommand\headrulewidth{0.4pt} % Size of the header rule
\renewcommand\footrulewidth{0.4pt} % Size of the footer rule

\setlength\parindent{0pt} % Removes all indentation from paragraphs

%--------------------------------------------------------------------------
%	CODE INCLUSION CONFIGURATION
%--------------------------------------------------------------------------
% This is the color used for comments
\definecolor{MyDarkGreen}{rgb}{0.0,0.4,0.0} 
\lstloadlanguages{Perl,Python} % Load Perl syntax for listings,
% for a list of other languages supported see:
%ftp://ftp.tex.ac.uk/tex-archive/macros/latex/contrib/listings/listings.pdf
\lstset{language=Perl, % Use Perl in this example
        frame=single, % Single frame around code
        basicstyle=\small\ttfamily, % Use small true type font
        keywordstyle=[1]\color{Blue}\bf, % Perl functions bold and blue
        keywordstyle=[2]\color{Purple}, % Perl function arguments purple
        % Custom functions underlined and blue
        keywordstyle=[3]\color{Blue}\underbar, 
        identifierstyle=, % Nothing special about identifiers
        % Comments small dark green courier font
        commentstyle=\usefont{T1}{pcr}{m}{sl}\color{MyDarkGreen}\small, 
        stringstyle=\color{Purple}, % Strings are purple
        showstringspaces=false, % Don't put marks in string spaces
        tabsize=5, % 5 spaces per tab
        %
        % Put standard Perl functions not included
        % in the default language here
        morekeywords={rand},
        %
        % Put Perl function parameters here
        morekeywords=[2]{on, off, interp},
        %
        % Put user defined functions here
        morekeywords=[3]{test},
       	%
        % Line continuation (...) like blue comment
        morecomment=[l][\color{Blue}]{...}, 
        numbers=left, % Line numbers on left
        firstnumber=1, % Line numbers start with line 1
        numberstyle=\tiny\color{Blue}, % Line numbers are blue and small
        stepnumber=5 % Line numbers go in steps of 5
}


\lstset{language=Python, % Use Python in this example
        frame=single, % Single frame around code
        basicstyle=\small\ttfamily, % Use small true type font
        keywordstyle=[1]\color{Blue}\bf, % Python functions bold and blue
        keywordstyle=[2]\color{Purple}, % Python function arguments purple
        % Custom functions underlined and blue
        keywordstyle=[3]\color{Blue}\underbar, 
        identifierstyle=, % Nothing special about identifiers
        % Comments small dark green courier font
        commentstyle=\usefont{T1}{pcr}{m}{sl}\color{MyDarkGreen}\small, 
        stringstyle=\color{Purple}, % Strings are purple
        showstringspaces=false, % Don't put marks in string spaces
        tabsize=3, % 5 spaces per tab
        %
        % Put standard Python functions not included in the
        % default language here
        morekeywords={rand},
        %
        % Put Python function parameters here
        morekeywords=[2]{on, off, interp},
        %
        % Put user defined functions here
        morekeywords=[3]{test},
       	%
        % Line continuation (...) like blue comment
        morecomment=[l][\color{Blue}]{...}, 
        numbers=left, % Line numbers on left
        firstnumber=1, % Line numbers start with line 1
        numberstyle=\tiny\color{Blue}, % Line numbers are blue and small
        stepnumber=5 % Line numbers go in steps of 5
}


% Creates a new command to include a perl script, the first
% parameter is the filename of the script (without .pl), the
% second parameter is the caption
\newcommand{\perlscript}[2]{
\begin{itemize}
\item[]\lstinputlisting[caption=#2,label=#1]{#1.pl}
\end{itemize}
}
\newcommand{\pythonscript}[2]{
\begin{itemize}
\item[]\lstinputlisting[caption=#2,label=#1]{#1}
\end{itemize}
}


%--------------------------------------------------------------------------
%	DOCUMENT STRUCTURE COMMANDS
%	Skip this unless you know what you're doing
%--------------------------------------------------------------------------

% Header and footer for when a page split occurs within a
% problem environment
\newcommand{\enterProblemHeader}[1]{
\nobreak\extramarks{#1}{#1 continued on next page\ldots}\nobreak
\nobreak\extramarks{#1 (continued)}{#1 continued on next page\ldots}\nobreak
}

% Header and footer for when a page split occurs between problem
% environments
\newcommand{\exitProblemHeader}[1]{
\nobreak\extramarks{#1 (continued)}{#1 continued on next page\ldots}\nobreak
\nobreak\extramarks{#1}{}\nobreak
}

\setcounter{secnumdepth}{0} % Removes default section numbers
% Creates a counter to keep track of the number of problems
\newcounter{homeworkProblemCounter} 

\newcommand{\homeworkProblemName}{}
\newenvironment{homeworkProblem}[1][Problem \arabic{homeworkProblemCounter}]{ % Makes a new environment called homeworkProblem which takes
   % 1 argument (custom name) but the default is "Problem #"
  % Increase counter for number of problems
  \stepcounter{homeworkProblemCounter}
  % Assign \homeworkProblemName the name of the problem
  \renewcommand{\homeworkProblemName}{#1}
  % Make a section in the document with the custom problem count
  \section{\homeworkProblemName}
  % Header and footer within the environment  
  \enterProblemHeader{\homeworkProblemName} 
}{
  % Header and footer after the environment
  \exitProblemHeader{\homeworkProblemName}
}
% Defines the problem answer command with the content as the only argument
 % Makes the box around the problem answer and puts the content inside
\newcommand{\problemAnswer}[1]{ 
\noindent\framebox[\columnwidth][c]{\begin{minipage}{0.98\columnwidth}#1\end{minipage}}
}

\newcommand{\homeworkSectionName}{}
\newenvironment{homeworkSection}[1]{ % New environment for sections within homework problems, takes 1 argument - the name of the section
\renewcommand{\homeworkSectionName}{#1} % Assign \homeworkSectionName to the name of the section from the environment argument
\subsection{\homeworkSectionName} % Make a subsection with the custom name of the subsection
\enterProblemHeader{\homeworkProblemName\ [\homeworkSectionName]} % Header and footer within the environment
}{
\enterProblemHeader{\homeworkProblemName} % Header and footer after the environment
}

%--------------------------------------------------------------------------
%	NAME AND CLASS SECTION
%--------------------------------------------------------------------------
\newcommand{\hmwkTitle}{Assignment\ 1} % Assignment title
\newcommand{\hmwkDueDate}{Tuesday,\ October\ 4,\ 2016} % Due date
\newcommand{\hmwkClass}{NUEN 647\\ Uncertainty Quantification for Nuclear Engineering} % Course/class
\newcommand{\hmwkClassTime}{Tu/Th 11:10am} % Class/lecture time
\newcommand{\hmwkClassInstructor}{Dr. McClarren} % Teacher/lecturer
\newcommand{\hmwkAuthorName}{Paul Mendoza} % Your name
\newcommand{\hmwkClassShort}{NUEN 647 UQ for Nuclear Engineering}
%--------------------------------------------------------------------------
%	TITLE PAGE
%--------------------------------------------------------------------------

\title{
\vspace{2in}
\textmd{\textbf{\hmwkClass\ \hmwkTitle}}\\
\normalsize\vspace{0.1in}\small{Due\ on\ \hmwkDueDate}\\
\vspace{0.1in}\large{\textit{\hmwkClassInstructor}}
\vspace{3in}
}

\author{\textbf{\hmwkAuthorName}}
\date{} % Insert date here if you want it to appear below your name

%--------------------------------------------------------------------------

\begin{document}

\maketitle

%--------------------------------------------------------------------------
%	TABLE OF CONTENTS
%--------------------------------------------------------------------------

%\setcounter{tocdepth}{1} % Uncomment this line if you don't want subsections listed in the ToC

\newpage
\tableofcontents
\newpage

Complete the exercises in the Chapter 2 notes. Be sure to include discussion of results
where appropriate. You may use any tools that are approrpriate to solving the problem.

%--------------------------------------------------------------------------
%	PROBLEM 1
%--------------------------------------------------------------------------

% To have just one problem per page, simply put a \clearpage after each problem

\begin{homeworkProblem}
  Show that the transformation in equation \ref{eq:1} results
  in a standard normal
  random variable by computing the mean and variance of z.

  \begin{equation} \label{eq:1}
    z = \frac{x-\mu}{\sigma}
  \end{equation}

  An important special case of the expectation value is the mean which
  is the expected value of $x$. It is often denoted as $\mu$,

  \begin{equation*}
    \mu=E[x]=\int_{-\infty}^\infty xf(x)dx
  \end{equation*}

  where $x$ is a realization of a random sample and $f(x)$ is
  the probability density function (PDF) for the random variable.
  For a normal distribution,

  \begin{equation*}
    f(x)=\frac{1}{\sigma\sqrt{2\pi}}e^{\frac{-(x-\mu)^2}{2\sigma^2}}
  \end{equation*}
  \problemAnswer{
  For the sake of the transformation, the value of z substitutes for
  $x$, the realization of a random sample (not the PDF because we are
  transforming that distribution). Therefore, the mean for
  z is:

  \begin{equation*}
    \mu_z=\int_{-\infty}^\infty \frac{x-\mu}{\sigma}
           \frac{1}{\sigma\sqrt{2\pi}}e^{\frac{-(x-\mu)^2}{2\sigma^2}}dx
  \end{equation*}

  If $u=(x-\mu)^2$ and $\frac{du}{2}=(x-\mu)dx$ (note that the limits
  change from $(-\infty,\infty)$ to $(\infty,\infty)$ - but that seems
  fishy to me so I will change it back after integration).

  \begin{equation*}
    \mu_z=\int_{\infty}^\infty
    \frac{1}{2\sigma^2\sqrt{2\pi}}e^{\frac{-u}{2\sigma^2}}du
    =\left|\frac{-1}{\sqrt{2\pi}}e^{\frac{-u}{2\sigma^2}}
      \right|_\infty^\infty
  \end{equation*}
  \begin{equation*}
    \mu_z
    =\left|\frac{-1}{\sqrt{2\pi}}e^{\frac{-(x-\mu)^2}{2\sigma^2}}
    \right|_{-\infty}^\infty=\frac{-1}{\sqrt{2\pi}}(e^{-\infty}-
    e^{-\infty}) =\boxed{0}
  \end{equation*}
  }
  \problemAnswer{
  The variance is defined as:

  \begin{equation*}
    Var(X)=E[(X-\mu)^2]
  \end{equation*}

  Substituting Eq. \ref{eq:1} for $X$, (but not for the pdf - I could
  be wrong about that)

  \begin{equation*}
    Var(X)=E[(\frac{x-\mu}{\sigma}-\mu)^2]=
    E\left[\left(\frac{x-\mu-\mu\sigma}{\sigma}\right)^2\right]=
    \frac{1}{\sigma^2}(E[x^2]-2\mu E[x]-2\mu\sigma E[x]+
    \mu^2E[1]+\mu^2\sigma^2+2\mu^2\sigma)
  \end{equation*}

  Noting that above it was proven that $E[x]=\mu$ and given that
  the definition of $E[1]=1$ and assuming that
  $E[x^2]=\sigma^2+\mu^2$ (will solve on next page)

  \begin{equation*}
    \frac{1}{\sigma^2}(\sigma^2+\mu^2-2\mu^2-2\mu^2\sigma+\mu^2
    +\mu^2\sigma^2+2\mu^2\sigma)=\frac{1}{\sigma^2}(\sigma^2+\mu^2\sigma^2)
    =\boxed{1+\mu^2=1}
  \end{equation*}
 
  This is assuming that $\mu=0$. Which was shown above.
  }
  \clearpage

  \begin{equation*}
    E[x^2]=\int_{-\infty}^{\infty} \frac{x^2}{\sigma\sqrt{2\pi}}
           e^{\frac{-(x-\mu)^2}{2\sigma^2}}
  \end{equation*}

  If $t=\frac{(x-\mu)}{\sqrt{2}\sigma}$ and $\sqrt{2}\sigma dt=dx$ and
  $x=t\sqrt{2}\sigma+\mu$
  then (limits of integration don't change)
  
  \begin{equation*}
    E[x^2]=\int_{-\infty}^{\infty} \frac{\left(t\sqrt{2}\sigma+\mu
      \right)^2}{\sqrt{\pi}}e^{-t^2}dt=
    \frac{1}{\sqrt{\pi}}\int_{-\infty}^{\infty}\left(2\sigma^
    2\left(t^2e^{-t^2}\right)+
    2\sqrt{2}\sigma\mu\left(te^{-t^2}\right)+\mu^2\left(e^{-t^2}\right)\right)
  \end{equation*}

  According to wolfram alpha

  \begin{equation*}
    \int_{-\infty}^\infty t^2e^{-t^2}=\frac{\sqrt{\pi}}{2}
  \end{equation*}

  \begin{equation*}
    \int_{-\infty}^\infty te^{-t^2}=0
  \end{equation*}

  \begin{equation*}
    \int_{-\infty}^\infty e^{-t^2}=\sqrt{\pi}
  \end{equation*}

  Which simplifies the above to $\sigma^2+\mu^2$.
%Listing \ref{Problem1/homework_example} shows a Perl script.

%\perlscript{Problem1/homework_example}{Sample Perl Script With Highlighting}

%\pythonscript{Problem1/Calculations}{Sample python script no .py}

%\pythonscript{Problem1/Functions.py}{Sample python script no .py}
%% \problemAnswer{
%% \begin{center}
%% \includegraphics[width=0.75\columnwidth]{Problem2/example_figure} % Example image
%% \end{center}

%% \lipsum[3-5]
%% }
  
\end{homeworkProblem}

\clearpage

%--------------------------------------------------------------------------
%	PROBLEM 2
%--------------------------------------------------------------------------

\begin{homeworkProblem}
%\lipsum[2]
  Consider the random variables $X\sim U(-1,1)$ and $Y\sim X^2$.
  Are these independent random variables? What is their covariance?
  \\~\\
  If two random variables, X and Y, are independent, they satisfy the
  following condition:
  \href{http://stattrek.com/random-variable/independence.aspx?Tutorial=AP}
       {$^{link}$}

  \begin{itemize}
    \item{$P(X|Y)=P(X)$, for all values of $X$ and $Y$.}
  \end{itemize}
  \problemAnswer{
  The PDF for X is:

  \begin{equation*}
    f_X(x)=\frac{1}{(1-(-1))}=0.5\ \ \ x \in [-1,1] 
  \end{equation*}

  The PDF for Y is:
  \href{http://math.stackexchange.com/questions/305997/does-the-square-of-uniform-distribution-have-density-function}{$^{link}$}

  \begin{equation*}
    f_Y(y)=\frac{1}{2\sqrt{y}}\ \ \ y \in [0,1] 
  \end{equation*}

  If the covariance is non zero, then these two variables are
  independant. The covariance of two random variables can be given
  by:
  \href{https://onlinecourses.science.psu.edu/stat414/node/111}{$^{link}$}
  
  \begin{equation*}
    \sigma_{XY}=E(XY)-\mu_X\mu_Y=\int_{-1}^{1}dx\int_{0}^{1}dy\ \
                xyf(x,y)-\mu_X\mu_Y
  \end{equation*}

  Because $\mu_X=0$ this reduces to

  \begin{equation*}
    \sigma_{XY}=E(XY)=\int_{-1}^{1}dx\int_{0}^{1}dy\ \
                xyf(x,y)
  \end{equation*}

  Where $f(x,y)$ is:
  
  \begin{equation*}
    f(x,y)=f(y|x)f_X(x)
  \end{equation*}

  From the definition of Y, $f(y|x)$ is 0 except when $y=x^2$.
  I think this would be.

  \begin{equation*}
    f(y|x)=\delta(y-x^2)
  \end{equation*}

  Which means,

  \begin{equation*}
    f(x,y)=0.5\delta(y-x^2)
  \end{equation*}
  
  I am not 100\% sure to do here, but after using wolfram,
  I think its 0.25.
  }
%% \problemAnswer{
%% \begin{center}
%% \includegraphics[width=0.75\columnwidth]{Problem2/example_figure} % Example image
%% \end{center}

%% \lipsum[3-5]
%% }
\end{homeworkProblem}

\clearpage

%--------------------------------------------------------------------------
%	PROBLEM 3
%--------------------------------------------------------------------------

\begin{homeworkProblem}
  Show that a general covariance matrix must be positive definite, i.e.
  $\vec{x}^T\Sigma\vec{x}>0$ for any vector $\vec{x}$ that is not
  all zeros.
  \\~\\
  \problemAnswer{
    Given that $\vec{Y}$ is a vector of random variables and
    $\vec{\mu}_Y$ is a vector of the mean values for the random
    variables found in $\vec{Y}$.
    
    \begin{align*}
      \vec{x}^T\Sigma\vec{x}&=
      \vec{x}^TE[(\vec{Y}-\vec{\mu}_Y)(\vec{Y}-\vec{\mu}_Y)^T]\vec{x}\\
      &=E[\vec{x}^T(\vec{Y}-\vec{\mu}_Y)(\vec{Y}-\vec{\mu}_Y)^T\vec{x}]
    \end{align*}
    The last step above puts a constant inside the expectation value
    integral. Notice
    \begin{equation*}
      \vec{x}^T(\vec{Y}-\vec{\mu}_Y)=(\vec{Y}-\vec{\mu}_Y)^T\vec{x}
    \end{equation*}
    and that both are scaler functions of the random variables. Therefore,
    \begin{align*}
      \vec{x}^T\Sigma\vec{x}&=E[(\vec{x}^T(\vec{Y}-\vec{\mu}_Y))^2]\\
      &=E[g(Y)^2]=\sigma_f^2
    \end{align*}
    The expectation value for a multivariate distribution is defined as
    \begin{equation*}
      E[g(Y)]=\int_{-\infty}^\infty dy_1
              \int_{-\infty}^\infty dy_2\ ...\int_{-\infty}^\infty dy_p
              \ g(y)f(y)
    \end{equation*}
    Where $f(y)$ is the multivariate PDF for the random variables
    of $\vec{Y}$.
    To prove that the covariance matrix is positive definite
    the above integral must be proved to be positive 
    with
    $g(x)=(\vec{x}^T(\vec{Y}-\vec{\mu}_Y))^2$. Explicitly, 
    \begin{align*}
      E[g(Y)]&=\int_{-\infty}^\infty dy_1
              \int_{-\infty}^\infty dy_2\ ...\int_{-\infty}^\infty dy_p
              \ (\vec{x}^T(\vec{Y}-\vec{\mu}_Y))^2f(y)\\
             &=\int_{-\infty}^\infty dy_1
              \int_{-\infty}^\infty dy_2\ ...\int_{-\infty}^\infty dy_p
              \ ((y_1-\mu_1)x_1+(y_2-\mu_2)x_2+\ ...\ +(y_p-\mu_p)x_p)^2f(y)\\
    \end{align*}
    
  }
\end{homeworkProblem}

\clearpage

%--------------------------------------------------------------------------
%	PROBLEM 4
%--------------------------------------------------------------------------

\begin{homeworkProblem}
  Use rejection sampling to sample from a Gamma random variable
  $X\sim \mathscr{G}(\alpha,\beta)$ where
  \begin{equation*}
    f(x)=\frac{\theta^{\alpha-1}e^{-\theta/\beta}}
         {\Gamma(\alpha)\beta^{-\alpha}}\ \ \ \alpha,\beta>0
  \end{equation*}

  Let $\alpha =1$ and $\beta=0.5$. From rejection sampling with a
  $N=10^4$, compute a rejection rate for the sampling procedure.
  Now draw a triangle around the function and do rejection sampling.
  Compare the rejection rate from the triangle versus the rectangle.
  You may consider that the PDF is zero if $f(x)<10^{-6}$.
  \\~\\
  Python script for rejection sampling.
  \pythonscript{Problem1/Calculations}{Python Script for problem}
%% \problemAnswer{
%% \begin{center}
%% \includegraphics[width=0.75\columnwidth]{Problem2/example_figure} % Example image
%% \end{center}

%% \lipsum[3-5]
%% }
\end{homeworkProblem}

\clearpage

%--------------------------------------------------------------------------
This is an example citation \cite{Tatro2013}.
\bibliography{references} 
\bibliographystyle{plain} 

\end{document}
