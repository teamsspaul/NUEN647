%%%%%%%%%%%%%%%%%%%%%%%%%%%%%%%%%%%%%%%%%
% Programming/Coding Assignment
% LaTeX Template
%
% This template has been downloaded from:
% http://www.latextemplates.com
%
% Original author:
% Ted Pavlic (http://www.tedpavlic.com)
%
% Note:
% The \lipsum[#] commands throughout this template generate dummy text
% to fill the template out. These commands should all be removed when 
% writing assignment content.
%
% This template uses a Perl script as an example snippet of code,
% most other languages are also usable. Configure them in the
% "CODE INCLUSION CONFIGURATION" section.
%%%%%%%%%%%%%%%%%%%%%%%%%%%%%%%%%%%%%%%%%

%-------------------------------------------------------------------------
%	PACKAGES AND OTHER DOCUMENT CONFIGURATIONS
%-------------------------------------------------------------------------

\documentclass{article}

\usepackage{fancyhdr} % Required for custom headers
\usepackage{lastpage} % Required to determine the last page for the footer
\usepackage{extramarks} % Required for headers and footers
\usepackage[usenames,dvipsnames]{color} % Required for custom colors
\usepackage{graphicx} % Required to insert images
\usepackage{listings} % Required for insertion of code
\usepackage{courier} % Required for the courier font
\usepackage{hyperref}
\usepackage{amsmath}
\usepackage{mathrsfs}
\usepackage{float}
\usepackage{enumitem}
% Margins
\topmargin=-0.45in
\evensidemargin=0in
\oddsidemargin=0in
\textwidth=6.5in
\textheight=9.0in
\headsep=0.25in

\linespread{1.1} % Line spacing

% Set up the header and footer
\pagestyle{fancy}
\lhead{\hmwkAuthorName} % Top left header
\chead{\hmwkClassShort\ (\hmwkClassInstructor)} % Top center head
%\rhead{\firstxmark} % Top right header
\rhead{\hmwkTitle}
\lfoot{\lastxmark} % Bottom left footer
\cfoot{} % Bottom center footer
% Bottom right footer
\rfoot{Page\ \thepage\ of\ \protect\pageref{LastPage}} 
\renewcommand\headrulewidth{0.4pt} % Size of the header rule
\renewcommand\footrulewidth{0.4pt} % Size of the footer rule

\setlength\parindent{0pt} % Removes all indentation from paragraphs

%--------------------------------------------------------------------------
%	CODE INCLUSION CONFIGURATION
%--------------------------------------------------------------------------
% This is the color used for comments
\definecolor{MyDarkGreen}{rgb}{0.0,0.4,0.0} 
\lstloadlanguages{Perl,Python} % Load Perl syntax for listings,
% for a list of other languages supported see:
%ftp://ftp.tex.ac.uk/tex-archive/macros/latex/contrib/listings/listings.pdf
\lstset{language=Perl, % Use Perl in this example
        frame=single, % Single frame around code
        basicstyle=\small\ttfamily, % Use small true type font
        keywordstyle=[1]\color{Blue}\bf, % Perl functions bold and blue
        keywordstyle=[2]\color{Purple}, % Perl function arguments purple
        % Custom functions underlined and blue
        keywordstyle=[3]\color{Blue}\underbar, 
        identifierstyle=, % Nothing special about identifiers
        % Comments small dark green courier font
        commentstyle=\usefont{T1}{pcr}{m}{sl}\color{MyDarkGreen}\small, 
        stringstyle=\color{Purple}, % Strings are purple
        showstringspaces=false, % Don't put marks in string spaces
        tabsize=5, % 5 spaces per tab
        %
        % Put standard Perl functions not included
        % in the default language here
        morekeywords={rand},
        %
        % Put Perl function parameters here
        morekeywords=[2]{on, off, interp},
        %
        % Put user defined functions here
        morekeywords=[3]{test},
       	%
        % Line continuation (...) like blue comment
        morecomment=[l][\color{Blue}]{...}, 
        numbers=left, % Line numbers on left
        firstnumber=1, % Line numbers start with line 1
        numberstyle=\tiny\color{Blue}, % Line numbers are blue and small
        stepnumber=5 % Line numbers go in steps of 5
}


\lstset{language=Python, % Use Python in this example
        frame=single, % Single frame around code
        basicstyle=\small\ttfamily, % Use small true type font
        keywordstyle=[1]\color{Blue}\bf, % Python functions bold and blue
        keywordstyle=[2]\color{Purple}, % Python function arguments purple
        % Custom functions underlined and blue
        keywordstyle=[3]\color{Blue}\underbar, 
        identifierstyle=, % Nothing special about identifiers
        % Comments small dark green courier font
        commentstyle=\usefont{T1}{pcr}{m}{sl}\color{MyDarkGreen}\small, 
        stringstyle=\color{Purple}, % Strings are purple
        showstringspaces=false, % Don't put marks in string spaces
        tabsize=3, % 5 spaces per tab
        %
        % Put standard Python functions not included in the
        % default language here
        morekeywords={rand},
        %
        % Put Python function parameters here
        morekeywords=[2]{on, off, interp},
        %
        % Put user defined functions here
        morekeywords=[3]{test},
       	%
        % Line continuation (...) like blue comment
        morecomment=[l][\color{Blue}]{...}, 
        numbers=left, % Line numbers on left
        firstnumber=1, % Line numbers start with line 1
        numberstyle=\tiny\color{Blue}, % Line numbers are blue and small
        stepnumber=5 % Line numbers go in steps of 5
}


% Creates a new command to include a perl script, the first
% parameter is the filename of the script (without .pl), the
% second parameter is the caption
\newcommand{\perlscript}[2]{
\begin{itemize}
\item[]\lstinputlisting[caption=#2,label=#1]{#1.pl}
\end{itemize}
}
\newcommand{\pythonscript}[2]{
\begin{itemize}
\item[]\lstinputlisting[caption=#2,label=#1]{#1}
\end{itemize}
}


%--------------------------------------------------------------------------
%	DOCUMENT STRUCTURE COMMANDS
%	Skip this unless you know what you're doing
%--------------------------------------------------------------------------

% Header and footer for when a page split occurs within a
% problem environment
\newcommand{\enterProblemHeader}[1]{
\nobreak\extramarks{#1}{#1 continued on next page\ldots}\nobreak
\nobreak\extramarks{#1 (continued)}{#1 continued on next page\ldots}\nobreak
}

% Header and footer for when a page split occurs between problem
% environments
\newcommand{\exitProblemHeader}[1]{
\nobreak\extramarks{#1 (continued)}{#1 continued on next page\ldots}\nobreak
\nobreak\extramarks{#1}{}\nobreak
}

\setcounter{secnumdepth}{0} % Removes default section numbers
% Creates a counter to keep track of the number of problems
\newcounter{homeworkProblemCounter} 

\newcommand{\homeworkProblemName}{}
\newenvironment{homeworkProblem}[1][Problem \arabic{homeworkProblemCounter}]{ % Makes a new environment called homeworkProblem which takes
   % 1 argument (custom name) but the default is "Problem #"
  % Increase counter for number of problems
  \stepcounter{homeworkProblemCounter}
  % Assign \homeworkProblemName the name of the problem
  \renewcommand{\homeworkProblemName}{#1}
  % Make a section in the document with the custom problem count
  \section{\homeworkProblemName}
  % Header and footer within the environment  
  \enterProblemHeader{\homeworkProblemName} 
}{
  % Header and footer after the environment
  \exitProblemHeader{\homeworkProblemName}
}
% Defines the problem answer command with the content as the only argument
 % Makes the box around the problem answer and puts the content inside
\newcommand{\problemAnswer}[1]{ 
\noindent\framebox[\columnwidth][c]{\begin{minipage}{0.98\columnwidth}#1\end{minipage}}
}

\newcommand{\homeworkSectionName}{}
\newenvironment{homeworkSection}[1]{ % New environment for sections within homework problems, takes 1 argument - the name of the section
\renewcommand{\homeworkSectionName}{#1} % Assign \homeworkSectionName to the name of the section from the environment argument
\subsection{\homeworkSectionName} % Make a subsection with the custom name of the subsection
\enterProblemHeader{\homeworkProblemName\ [\homeworkSectionName]} % Header and footer within the environment
}{
\enterProblemHeader{\homeworkProblemName} % Header and footer after the environment
}

%--------------------------------------------------------------------------
%	NAME AND CLASS SECTION
%--------------------------------------------------------------------------
\newcommand{\hmwkTitle}{Assignment\ 1} % Assignment title
\newcommand{\hmwkDueDate}{Tuesday,\ October\ 4,\ 2016} % Due date
\newcommand{\hmwkClass}{NUEN 647\\ Uncertainty Quantification for Nuclear Engineering} % Course/class
\newcommand{\hmwkClassTime}{Tu/Th 11:10am} % Class/lecture time
\newcommand{\hmwkClassInstructor}{Dr. McClarren} % Teacher/lecturer
\newcommand{\hmwkAuthorName}{Paul Mendoza} % Your name
\newcommand{\hmwkClassShort}{NUEN 647 UQ for Nuclear Engineering}
%--------------------------------------------------------------------------
%	TITLE PAGE
%--------------------------------------------------------------------------

\title{
\vspace{2in}
\textmd{\textbf{\hmwkClass\ \hmwkTitle}}\\
\normalsize\vspace{0.1in}\small{Due\ on\ \hmwkDueDate}\\
\vspace{0.1in}\large{\textit{\hmwkClassInstructor}}
\vspace{3in}
}

\author{\textbf{\hmwkAuthorName}}
\date{} % Insert date here if you want it to appear below your name

%--------------------------------------------------------------------------

\begin{document}

\maketitle

%--------------------------------------------------------------------------
%	TABLE OF CONTENTS
%--------------------------------------------------------------------------

%\setcounter{tocdepth}{1} % Uncomment this line if you don't want subsections listed in the ToC

\newpage
\tableofcontents
\newpage

Complete the exercises in the Chapter 2 notes. Be sure to include discussion of results
where appropriate. You may use any tools that are approrpriate to solving the problem.

%--------------------------------------------------------------------------
%	PROBLEM 1
%--------------------------------------------------------------------------

% To have just one problem per page, simply put a \clearpage after each problem

\begin{homeworkProblem}
  Show that the transformation in equation \ref{eq:1} results
  in a standard normal
  random variable by computing the mean and variance of z.

  \begin{equation} \label{eq:1}
    z = \frac{x-\mu}{\sigma}
  \end{equation}

  An important special case of the expectation value is the mean which
  is the expected value of $x$. It is often denoted as $\mu$,

  \begin{equation*}
    \mu=E[x]=\int_{-\infty}^\infty xf(x)dx
  \end{equation*}

  where $x$ is a realization of a random sample and $f(x)$ is
  the probability density function (PDF) for the random variable.
  For a normal distribution,

  \begin{equation*}
    f(x)=\frac{1}{\sigma\sqrt{2\pi}}e^{\frac{-(x-\mu)^2}{2\sigma^2}}
  \end{equation*}
  \problemAnswer{
  For the sake of the transformation, the value of z substitutes for
  $x$, the realization of a random sample (not the PDF because we are
  transforming that distribution). Therefore, the mean for
  z is:

  \begin{equation*}
    \mu_z=\int_{-\infty}^\infty \frac{x-\mu}{\sigma}
           \frac{1}{\sigma\sqrt{2\pi}}e^{\frac{-(x-\mu)^2}{2\sigma^2}}dx
  \end{equation*}

  If $u=(x-\mu)^2$ and $\frac{du}{2}=(x-\mu)dx$ (note that the limits
  change from $(-\infty,\infty)$ to $(\infty,\infty)$ - but that seems
  fishy to me so I will change it back after integration).

  \begin{equation*}
    \mu_z=\int_{\infty}^\infty
    \frac{1}{2\sigma^2\sqrt{2\pi}}e^{\frac{-u}{2\sigma^2}}du
    =\left|\frac{-1}{\sqrt{2\pi}}e^{\frac{-u}{2\sigma^2}}
      \right|_\infty^\infty
  \end{equation*}
  \begin{equation*}
    \mu_z
    =\left|\frac{-1}{\sqrt{2\pi}}e^{\frac{-(x-\mu)^2}{2\sigma^2}}
    \right|_{-\infty}^\infty=\frac{-1}{\sqrt{2\pi}}(e^{-\infty}-
    e^{-\infty}) =\boxed{0}
  \end{equation*}
  }
  \problemAnswer{
  The variance is defined as:

  \begin{equation*}
    Var(X)=E[(X-\mu_X)^2]
  \end{equation*}

  Substituting Eq. \ref{eq:1} for $X$, (but not for the pdf)

  \begin{equation*}
    Var(Z)=E[(\frac{x-\mu_X}{\sigma_X}-\mu_Z)^2]=
    E\left[\left(\frac{x-\mu_X}{\sigma_X}\right)^2\right]=
    \frac{1}{\sigma_X^2}(E[x^2]-2\mu_XE[x]+
    \mu^2E[1])
  \end{equation*}

  Noting that above it was proven that $E[x]=\mu_X$ and given that
  the definition of $E[1]=1$ and assuming that
  $E[x^2]=\sigma_X^2+\mu_X^2$ (will solve on next page)

  \begin{equation*}
    \frac{1}{\sigma_X^2}(\sigma_X^2+\mu_X^2-2\mu_X^2+\mu_X^2
    )=\boxed{1}
  \end{equation*}
 
  }
  \clearpage

  \begin{equation*}
    E[x^2]=\int_{-\infty}^{\infty} \frac{x^2}{\sigma\sqrt{2\pi}}
           e^{\frac{-(x-\mu)^2}{2\sigma^2}}
  \end{equation*}

  If $t=\frac{(x-\mu)}{\sqrt{2}\sigma}$ and $\sqrt{2}\sigma dt=dx$ and
  $x=t\sqrt{2}\sigma+\mu$
  then (limits of integration don't change)
  
  \begin{equation*}
    E[x^2]=\int_{-\infty}^{\infty} \frac{\left(t\sqrt{2}\sigma+\mu
      \right)^2}{\sqrt{\pi}}e^{-t^2}dt=
    \frac{1}{\sqrt{\pi}}\int_{-\infty}^{\infty}\left(2\sigma^
    2\left(t^2e^{-t^2}\right)+
    2\sqrt{2}\sigma\mu\left(te^{-t^2}\right)+\mu^2\left(e^{-t^2}\right)\right)
  \end{equation*}

  According to wolfram alpha

  \begin{equation*}
    \int_{-\infty}^\infty t^2e^{-t^2}=\frac{\sqrt{\pi}}{2}
  \end{equation*}

  \begin{equation*}
    \int_{-\infty}^\infty te^{-t^2}=0
  \end{equation*}

  \begin{equation*}
    \int_{-\infty}^\infty e^{-t^2}=\sqrt{\pi}
  \end{equation*}

  Which simplifies the above to $\sigma^2+\mu^2$.
%Listing \ref{Problem1/homework_example} shows a Perl script.

%\perlscript{Problem1/homework_example}{Sample Perl Script With Highlighting}

%\pythonscript{Problem1/Calculations}{Sample python script no .py}

%\pythonscript{Problem1/Functions.py}{Sample python script no .py}
%% \problemAnswer{
%% \begin{center}
%% \includegraphics[width=0.75\columnwidth]{Problem2/example_figure} % Example image
%% \end{center}

%% \lipsum[3-5]
%% }
  
\end{homeworkProblem}

\clearpage

%--------------------------------------------------------------------------
%	PROBLEM 2
%--------------------------------------------------------------------------

\begin{homeworkProblem}
%\lipsum[2]
  Consider the random variables $X\sim U(-1,1)$ and $Y\sim X^2$.
  Are these independent random variables? What is their covariance?
  \\~\\
  \problemAnswer{
  \textbf{Marginal and Joint PDFs}\\~\\
  The PDF for X is:

  \begin{equation*}
    f_X(x)=\frac{1}{(1-(-1))}=0.5\ \ \ x \in [-1,1] 
  \end{equation*}

  The PDF for Y is:
  \href{http://math.stackexchange.com/questions/305997/does-the-square-of-uniform-distribution-have-density-function}{$^{link}$}
  
  \begin{equation*}
    f_Y(y)=\frac{1}{2\sqrt{y}}\ \ \ y \in [0,1] 
  \end{equation*}

  Without using the handy reference, this may be derived from the
  joint PDF, $f(x,y)$, defined as (between McClarrens Eq. 2.30 and 2.31):
  
  \begin{equation*}
    f(x,y)=f(y|x)f_X(x)
  \end{equation*}

  From the definition of Y, $f(y|x)$ is 0 except when $y=x^2$.
  I think this would be.

  \begin{equation*}
    f(y|x)=\delta(y-x^2)
  \end{equation*}

  Which means,

  \begin{equation*}
    f(x,y)=0.5\delta(y-x^2)
  \end{equation*}

  To calculate the PDF for Y (f(y)), we need to integrate
  over all other variables (in this case, X).

  \begin{equation*}
    f(y)=\int_{-1}^{1} f(x,y) dx=\int_{-1}^{1} 0.5\delta(y-x^2) dx
  \end{equation*}

  Wolfram alpha tells me the answer is,

  \begin{equation*}
    f_Y(y)=\frac{1}{2\sqrt{y}}\ \ \ y \in [0,1] 
  \end{equation*}

  The same as above. 
  }
  \problemAnswer{
    \textbf{Independance}\\~\\
    If two random variables, X and Y, are independent, they satisfy the
    following condition:
   \href{http://stattrek.com/random-variable/independence.aspx?Tutorial=AP}
       {$^{link}$}

    \begin{itemize}
      \item{$P(Y|X)=P(Y)$, for all values of $X$ and $Y$.}
    \end{itemize}
    Because $P(Y|X)=\delta(y-xY2)\neq P(Y)=\frac{1}{2\sqrt{y}}$
    (at least not for ALL values of x and y),
    these two variables are $\boxed{dependent}$.
  }
  \problemAnswer{
    \textbf{Covariance}\\~\\
    The covariance for two random variables is defined as:
    \begin{equation*}
      \sigma_{XY}=E[(x-\mu_X)(y-\mu_Y)]
    \end{equation*}
  This simplifies down to:
  %\href{https://onlinecourses.science.psu.edu/stat414/node/111}{$^{link}$}
  
  \begin{equation*}
    \sigma_{XY}=E(XY)-\mu_X\mu_Y=\int_{-1}^{1}dx\int_{0}^{1}dy\ \
                xyf(x,y)-\mu_X\mu_Y
  \end{equation*}

  Because $\mu_X=0$ this reduces to

  \begin{align*}
    \sigma_{XY}=E(XY)&=\int_{-1}^{1}dx\int_{0}^{1}dy\ \
    xyf(x,y)\\
    &=\int_{-1}^{1}dx\int_{0}^{1}dy\ \
    xy0.5\delta(y-x^2)\\
    &=\int_{-1}^{1}dx\ \ 0.5x^3dx\\
    &=\boxed{0}
  \end{align*}
  
  Wolfram alpha gave the step between the second and third line.
  These variables are dependant, but have a zero covariance. 
  }
%% \problemAnswer{
%% \begin{center}
%% \includegraphics[width=0.75\columnwidth]{Problem2/example_figure} % Example image
%% \end{center}

%% \lipsum[3-5]
%% }
\end{homeworkProblem}

\clearpage

%--------------------------------------------------------------------------
%	PROBLEM 3
%--------------------------------------------------------------------------

\begin{homeworkProblem}
  Show that a general covariance matrix must be positive definite, i.e.
  $\vec{x}^T\Sigma\vec{x}>0$ for any vector $\vec{x}$ that is not
  all zeros.
  \\~\\
  \problemAnswer{
    Given that $\vec{Y}$ is a vector of random variables and
    $\vec{\mu}_Y$ is a vector of the mean values for the random
    variables found in $\vec{Y}$.
    
    \begin{align*}
      \vec{x}^T\Sigma\vec{x}&=
      \vec{x}^TE[(\vec{Y}-\vec{\mu}_Y)(\vec{Y}-\vec{\mu}_Y)^T]\vec{x}\\
      &=E[\vec{x}^T(\vec{Y}-\vec{\mu}_Y)(\vec{Y}-\vec{\mu}_Y)^T\vec{x}]
    \end{align*}
    The last step above puts a constant inside the expectation value
    integral. Notice
    \begin{equation*}
      \vec{x}^T(\vec{Y}-\vec{\mu}_Y)=(\vec{Y}-\vec{\mu}_Y)^T\vec{x}
    \end{equation*}
    and that both are scaler functions of the random variables. Therefore,
    \begin{align*}
      \vec{x}^T\Sigma\vec{x}&=E[(\vec{x}^T(\vec{Y}-\vec{\mu}_Y))^2]\\
      &=E[g(\vec{Y})^2]=\sigma_f^2
    \end{align*}
    The expectation value for a multivariate distribution is defined as
    \begin{equation*}
      E[g(\vec{Y})]=\int_{-\infty}^\infty dy_1
              \int_{-\infty}^\infty dy_2\ ...\int_{-\infty}^\infty dy_p
              \ g(\vec{y})f(\vec{y})
    \end{equation*}
    Where $f(\vec{y})$ is the multivariate PDF for the random variables
    of $\vec{Y}$. If a number of samples is given, rather than functions
    that can be integrated, the expectation value for a multivariate
    distribution is defined as:
    \begin{equation*}
      E[g(\vec{Y})]=SC
    \end{equation*}
    To prove that the covariance matrix is positive definite
    the above integral must be proved to be positive 
    with
    $g(x)=(\vec{x}^T(\vec{Y}-\vec{\mu}_Y))^2$. Explicitly, 
    \begin{align*}
      E[g(Y)]&=\int_{-\infty}^\infty dy_1
              \int_{-\infty}^\infty dy_2\ ...\int_{-\infty}^\infty dy_p
              \ (\vec{x}^T(\vec{Y}-\vec{\mu}_Y))^2f(y)\\
             &=\int_{-\infty}^\infty dy_1
              \int_{-\infty}^\infty dy_2\ ...\int_{-\infty}^\infty dy_p
              \ ((y_1-\mu_1)x_1+(y_2-\mu_2)x_2+\ ...\ +(y_p-\mu_p)x_p)^2f(y)\\
    \end{align*}
    
  }
\end{homeworkProblem}

\clearpage

%--------------------------------------------------------------------------
%	PROBLEM 4
%--------------------------------------------------------------------------

\begin{homeworkProblem}
  Use rejection sampling to sample from a Gamma random variable
  $X\sim \mathscr{G}(\alpha,\beta)$ where
  \begin{equation*}
    f(x)=\frac{\theta^{\alpha-1}e^{-\theta\beta}}
         {\Gamma(\alpha)\beta^{-\alpha}}\ \ \ \alpha,\beta>0
  \end{equation*}

  Let $\alpha =1$ and $\beta=0.5$. From rejection sampling with a
  $N=10^4$, compute a rejection rate for the sampling procedure.
  Now draw a triangle around the function and do rejection sampling.
  Compare the rejection rate from the triangle versus the rectangle.
  You may consider that the PDF is zero if $f(x)<10^{-6}$.
  \\~\\
  Python script for rejection sampling.
  \pythonscript{Problem4/Calculations}{Python Script for problem}
  \vspace{-7mm}
  \begin{figure}[H]
  \begin{center}
    \includegraphics[width=0.77\columnwidth]{Problem4/P4F1.pdf}
    \vspace{-5mm}
    \caption{Square (top) and triangular (bottom) rejection sampling
             for the Gamma random variable.}
    \label{fig:P4F1}
  \end{center}
  \end{figure}
  \vspace{-2mm}
  \problemAnswer{
    The rejection rate for the square is $\boxed{92.65\%}$.\\
    The rejection rate for the triangle is $\boxed{85.04\%}$.\\~\\
    The acceptance rate about doubled from the square to the triangle
    ($(1-0.9265)*2=0.147\approx0.1496=(1-0.8504)$). This is what
    is expected because we cut the sampling area in half.
    This could also be used to verify the PDF is properly normalized.
    $0.1496*0.5*26.245*1/2=0.98$.
  }
%% \problemAnswer{
%% \begin{center}
%% \includegraphics[width=0.75\columnwidth]{Problem2/example_figure} % Example image
%% \end{center}

%% \lipsum[3-5]
%% }
\end{homeworkProblem}

\clearpage


%--------------------------------------------------------------------------
%	PROBLEM 5
%--------------------------------------------------------------------------

\begin{homeworkProblem}
  Consider a random variable, $X>0$, that has it's logarithm distributed
  by a normal distribution with mean $\mu=0$ and variance $\sigma^2=1$.
  Such a distribution is called a log-normal distribution. Compute this
  distribution's a) mean, b) variance, c) median, d) mode, e) skew,
  and d) kurtosis.
  \\~\\
  The PDF for the log-normal distribution, found on wikipedia, is:
  \begin{equation*}
    f(X)=\frac{1}{x\sigma\sqrt{2\pi}}e^{-\frac{\left(ln(x)-\mu\right)^2}{2\sigma^2}}
  \end{equation*}
  For a standard log-normal distribution this is
  \begin{equation*}
    f(X)=\frac{1}{x\sqrt{2\pi}}
    e^{-\frac{ln(x)^2}{2}}
  \end{equation*}
  Which simplifies to:
  \begin{align*}
    f(X)&=\frac{1}{x\sqrt{2\pi}}
    \left(e^{-ln(x)}\right)^{\frac{ln(x)}{2}}\\
    &=\frac{1}{x\sqrt{2\pi}}
    \left(\frac{1}{x}\right)^{\frac{ln(x)}{2}}
  \end{align*}
  \\~\\
  \problemAnswer{
    \textbf{a) mean}\\~\\
    The mean, $\mu$, is defined as:
    
    \begin{equation*}
      \mu=E[x]=\int_{-\infty}^\infty xf(x)dx
    \end{equation*}
    Using the log-normal distribution the mean is:
    \begin{align*}
      \mu=E[x]&=\int_{0}^\infty
      \frac{1}{\sqrt{2\pi}}\left(\frac{1}{x}\right)
      ^{\frac{ln(x)}{2}}dx\\
      &=\frac{1}{\sqrt{2\pi}}\int_{0}^\infty
      \left(\frac{1}{x}\right)
      ^{\frac{ln(x)}{2}}dx\\
    \end{align*}
    Wolfram alpha says
    \begin{equation*}
      \int_{0}^\infty
      \left(\frac{1}{x}\right)
      ^{\frac{ln(x)}{2}}dx=\sqrt{2e\pi}
    \end{equation*}
    Therefore the answer is $\boxed{\mu=\sqrt{e}}$
  }
  \problemAnswer{
    \textbf{b) Variance}\\~\\
    Variance, $\sigma^2$, is defined as:
    
    \begin{align*}
      \sigma^2&=E[(X-\mu)^2]\\
         &=E[X^2]-2E[\mu X]+E[\mu^2]\\
         &=E[X^2]-\mu^2\\
         &=\int_{-\infty}^\infty x^2f(x)dx-\mu^2
    \end{align*}
    Using the log-normal distribution the variance is:
    \begin{align*}
      \sigma^2=E[x^2]-\mu^2&=\int_{0}^\infty
      \frac{x}{\sqrt{2\pi}}\left(\frac{1}{x}\right)
      ^{\frac{ln(x)}{2}}dx-\mu^2\\
            &=\frac{1}{\sqrt{2\pi}}\int_{0}^\infty
      x\left(\frac{1}{x}\right)
      ^{\frac{ln(x)}{2}}dx-\mu^2
    \end{align*}
    Wolfram alpha says
    \begin{equation*}
      \int_{0}^\infty
      x\left(\frac{1}{x}\right)
      ^{\frac{ln(x)}{2}}dx=e^2\sqrt{2\pi}
    \end{equation*}
    Therefore the answer is $\boxed{\sigma^2=e(e-1)}$
  }
  \problemAnswer{
    \textbf{c) Median}\\~\\
    The median is defined as the point where the CDF
    is equal to one-half. The CDF, in terms of the PDF is:
    \begin{equation*}
      F_X(x)=\int_{-\infty}^{x}f(x')dx'
    \end{equation*}
    For the log-normal distribution
    \begin{align*}
      F_X(x)&=\int_{0}^{x}\frac{1}{x'\sigma\sqrt{2\pi}}
      e^{-\frac{(ln(x')-\mu)^2}{2\sigma^2}}dx'\\
      &=\frac{1}{2}+\frac{1}{2}erf\left[\frac{ln(x)-\mu}{\sqrt{2}\sigma}
      \right]
    \end{align*}
    If $\sigma=1$ and $\mu=0$,
    \begin{equation*}
      F_X(x)=\frac{1}{2}+
      \frac{1}{2}erf\left[\frac{ln(x)}{\sqrt{2}}\right]
    \end{equation*}
    If $F_X(x)=\frac{1}{2}$ then
    \begin{align*}
      erf^{-1}[0]\sqrt{2}&=ln(x)\\
      e^{erf^{-1}[0]\sqrt{2}}&=x
    \end{align*}
    $erf^{-1}[0]=0$, because wolfram says so...\\~\\
    Therefore the answer is $\boxed{Median=1}$
  }
  \problemAnswer{
    \textbf{d) Mode}\\~\\
    The mode is the point where the PDF takes its maximum value,
    the most likely value of the distribution. This could also
    be the point where its derivative goes to 0.
    \begin{equation*}
      f(X)=\frac{1}{x\sqrt{2\pi}}
      e^{-\frac{ln(x)^2}{2}}
    \end{equation*}
    Derivatives:
    \begin{equation*}
      e^{-\frac{ln(x)^2}{2}}\frac{d}{dx}=
      \frac{-ln(x)}{x}e^{-\frac{ln(x)^2}{2}}
    \end{equation*}
    \begin{equation*}
      \frac{1}{x\sqrt{2\pi}}\frac{d}{dx}=
      -\frac{1}{x^2\sqrt{2\pi}}
    \end{equation*}
    Combining:
    \begin{equation*}
      f'(X)=\left(\frac{-ln(x)-1}{x^2\sqrt{2\pi}}\right)
      e^{-\frac{ln(x)^2}{2}}
    \end{equation*}
    Setting Equal to Zero
    \begin{align*}
      0&=\left(\frac{-ln(x)-1}{x^2\sqrt{2\pi}}\right)
      e^{-\frac{ln(x)^2}{2}}\\
      0&=(-ln(x)-1)
      e^{-\frac{ln(x)^2}{2}}\\
      0&=(-ln(x)-1)\\
      -1&=ln(x)\\
      x&\boxed{=e^{-1}}
    \end{align*}
  }
  \problemAnswer{
    \textbf{e) Skew}\\~\\
    The skewness, $\gamma_1$, is related to the third moment of
    $f(x)$, that is the expected value of $X^3$.
    \begin{equation*}
      \gamma_1=\frac{E[(X-\mu)^3]}{Var(X)^{3/2}}
    \end{equation*}
    Earlier it was proven that $Var(X)=e^2-e$. Expanding the
    numerator...
    \begin{align*}
      (X-\mu)^3&=(X-\mu)(X-\mu)(X-\mu)\\
      &=X^3-3X^2\mu+3X\mu^2-\mu^3\\
      E[(X-\mu)^3]&=E[X^3]-3E[X^2]\mu+2\mu^3
    \end{align*}
    The last two terms have $\mu^3$ in them because $E[X]=\mu$.
    Recall, that earlier, we discovered $E[X^2]=e^2$ and that
    $\mu=\sqrt{e}$. Plugging in
    what we know,
    \begin{equation*}
      \gamma_1=\frac{E[X^3]-3e^2\sqrt{e}+2e^{3/2}}{(e^2-e)^{3/2}}
    \end{equation*}
    To determine $E[X^3]$,
    \begin{align*}
      E[X^3]&=\int_{0}^\infty
      \frac{x^2}{\sqrt{2\pi}}\left(\frac{1}{x}\right)
      ^{\frac{ln(x)}{2}}dx\\
            &=\frac{1}{\sqrt{2\pi}}\int_{0}^\infty
      x^2\left(\frac{1}{x}\right)
      ^{\frac{ln(x)}{2}}dx
    \end{align*}
    Wolfram alpha says
    \begin{equation*}
      \int_{0}^\infty
      x^2\left(\frac{1}{x}\right)^
      {\frac{ln(x)}{2}}dx=e^{9/2}\sqrt{2\pi}
    \end{equation*}
    Which Means:
    \begin{equation*}
      E[X^3]=e^{9/2}
    \end{equation*}
    Plugging this in,
    \begin{equation*}
      \gamma_1=\boxed{\frac{e^{9/2}-3e^{5/2}+2e^{3/2}}{(e^2-e)^{3/2}}
        =(e+2)\sqrt{e-1}=6.1849}
    \end{equation*}
    The standard log-normal distribution has a positive skew,
    which means that the distribution goes to zero more slowly
    to the right of the mean (look up log-normal distribution
    on wikipedia, and you'll be able to see that is the case).
  }

  \problemAnswer{
    \textbf{f) Kurtosis}\\~\\
    The excess kurtosis is a measure of ``tail fatness'' for
    a distribution, related to the fourth moment of a
    random variable's PDF:
    \begin{equation*}
      \gamma_1=\frac{E[(X-\mu)^4]}{Var(X)^{2}}-3=
               \frac{E[(X-\mu)^4]}{\sigma^{4}}-3
    \end{equation*}
    It should be noted that these $\mu$'s and $\sigma$'s are the
    $\mu$'s and $\sigma$'s from the log-normal distribution,
    not the $\mu$'s and $\sigma$'s from the normal distribution
    with $\mu=0$ and $\sigma=1$ by which the logarithm was distributed.
    Earlier it was proven that $\sigma^2=e^2-e$, $E[X]=\mu=\sqrt{e}$,
    $E[X^2]=e^2$, and that $E[X^3]=e^{9/2}$. Expanding the
    numerator...
    \begin{align*}
      (X-\mu)^4&=(X-\mu)(X-\mu)(X-\mu)(X-\mu)\\
      &=X^4-4X^3\mu+6X^2\mu^2-4X\mu^3+\mu^4
    \end{align*}
    To determine $E[X^4]$,
    \begin{align*}
      E[X^4]&=\int_{0}^\infty
      \frac{x^3}{\sqrt{2\pi}}\left(\frac{1}{x}\right)
      ^{\frac{ln(x)}{2}}dx\\
            &=\frac{1}{\sqrt{2\pi}}\int_{0}^\infty
      x^3\left(\frac{1}{x}\right)
      ^{\frac{ln(x)}{2}}dx
    \end{align*}
    Wolfram alpha says
    \begin{equation*}
      \int_{0}^\infty
      x^3\left(\frac{1}{x}\right)^
      {\frac{ln(x)}{2}}dx=e^{8}\sqrt{2\pi}
    \end{equation*}
    Which Means:
    \begin{equation*}
      E[X^4]=e^{8}
    \end{equation*}
    Plugging everything in:
    \begin{align*}
      \gamma_1&=\frac{E[(X-\mu)^4]}{\sigma^{4}}-3\\
      &=\frac{E[X^4]-4E[X^3]\mu+6E[X^2]\mu^2-4E[X]\mu^3+
        \mu^4]}{\sigma^{4}}-3\\
      &=\boxed{\frac{e^{8}-4e^{9/2}\sqrt{e}+6e^2e-4\sqrt{e}e^{3/2}+
        e^{2}]}{(e^2-e)^2}-3}\\
      &=\boxed{e^4+2e^3+3e^2-6=110.94}
    \end{align*}
    Because the kurtosis is positive, we can deduce that
    the standard log-normal distribution has heavier tails than
    a normal distribution. Meaning that the PDF doesn't approach
    zero as quickly as for a normal distribution.
  }  
\end{homeworkProblem}

\clearpage


%--------------------------------------------------------------------------
%	PROBLEM 6
%--------------------------------------------------------------------------

\begin{homeworkProblem}
  (Monte Hall Problem) You are on a game show and are presented with
  three doors from which to choose. One of the doors contains a prize
  and the other two have nothing. You pick a door (say door 1), and
  then the host opens another door (say door 3), and asks you if you
  want to switch to door number 2. What should you do?

  \begin{enumerate}[label=\alph*]
  \item{Using Bayes' theorem give the probability of
    winning if you switch.}
  \item{Write a simulation code to show that by randomly
    assigning a prize to a door, then opening either
    door 2 or 3 depending on which has the prize, and then
    either switching or not. Compute the likelihood of winning
    if you stick, versus the likelihood of winning if you switch.}
  \end{enumerate}

  \problemAnswer{
    \textbf{Bayes' theorem}
    \begin{equation*}
      f(x|y)=\frac{f(y|x)f_X(x)}{f_Y(y)}
    \end{equation*}
    \begin{itemize}
    \item{Define $f(x|y)$ as $f(P2|D3)$, the probability of
    the prize being behind door 2, given that door 3 was opened
    to show nothing.}
    \item{Define $f(y|x)$ as $f(D3|P2)$, the probability of
      opening door 3 to reveal nothing, given that the prize
      is behind door 2. This probability is 1, because the host will
      either open door 2 or door 3 (we picked door 1). Since
      good ol host will not open the door with the prize, he
      will open door 3.}
    \item{Define $f_X(x)$ as $f(P2)$, the probability of finding
      the prize behind door 2. This probability is 1/3 because,
      all things being equal, the prize is equally likely to be
      found behind any of the three doors}
    \item{Define $f_Y(y)$ as $f(D3)$, the probabilty the host
      openings  door 3 to show nothing. There are three options for this.
      First, if the prize is behind door 1, the host will open
      door 3 half the time. Second, if the prize is behind door
      2, the host will open door 3 every time. Third, if the prize
      is behind door 3, the host will never open door 3. This equates
      to:
      \begin{equation*}
        f(D3)=\frac{1}{3}\cdot\frac{1}{2}+\frac{1}{3}\cdot1+\frac{1}{3}
        \cdot0=\frac{1}{2}
      \end{equation*}
      }
    \begin{equation*}
      f(P2|D3)=\frac{f(D3|P2)f(P2)}{f(D3)}=
      \frac{1\cdot\frac{1}{3}}{\frac{1}{2}}=\boxed{\frac{2}{3}}
    \end{equation*}
    \end{itemize}  
  }
  \\
  
  Python script computing the likelihood of winning.
  \pythonscript{Problem6/Calculations}{Python Script for problem}

  \problemAnswer{
    Output of code:\\~\\
    You win 33\% of the time if you stay.\\
    You win 67\% of the time if you switch.
  }
\end{homeworkProblem}

\clearpage

%--------------------------------------------------------------------------
%	PROBLEM 7
%--------------------------------------------------------------------------

\begin{homeworkProblem}
  Consider a variable $Y$ distributed by a normal distribution
  with mean given by $\theta$:
  \begin{equation*}
    f(f|\theta)=\frac{1}{\sigma\sqrt{2\pi}}
    e^{\left(-\frac{(y-\theta)^2}{2\sigma^2}\right)}
  \end{equation*}
  Now consider $\theta$ to be a random variable as well, and $\sigma$
  to be a known constant. Then say $\theta$ is normally distributed,
  with mean $\mu$ and variance $\tau^2$ to give
  \begin{equation*}
    \pi(\theta)=\frac{1}{\tau\sqrt{2\pi}}
        e^{\left(-\frac{(\theta-\mu)^2}{2\tau^2}\right)}
  \end{equation*}
  The parameters $\mu$ and $\tau$ are called hyperparameters. Using
  Bayes' theorem find $p(\theta|y)$, and show that it is a normal
  distribution.
\end{homeworkProblem}


\clearpage


%--------------------------------------------------------------------------
%	PROBLEM 8
%--------------------------------------------------------------------------

\begin{homeworkProblem}
  Suppose that X is the number of people arriving at a particular
  tavern during a given hour. This type of arrival process is naturally
  described by a Poisson process:
  \begin{equation*}
    f(x|\theta)=\frac{e^{-\theta}\theta^{x}}{x!},\ \ \ \ x\in\{0,1,2,...\},
    \ \ \theta>0.
  \end{equation*}
  We then say that our prior distribution of $\theta$ is a Gamma
  distribution.
  \begin{equation*}
    \pi(\theta)=\frac{\theta^{\alpha-1}e^{-\theta\beta}}{\Gamma(\alpha)
    \beta^{-\alpha}},\ \ \ \ \ \theta,\alpha,\beta>0.
  \end{equation*}
  Therefore, we say that $\theta\sim G(\alpha,\beta)$.
  \begin{itemize}
  \item{Show using Bayes' theorem that the posterior distribution for
    $\theta$ given $x$ is proportional to a Gamma distribution.}
  \item{Suppose you observe 42 people arriving in one hour, and the
    prior distribution has $\alpha=5$ and $\beta=6$. Generate samples
    from the posterior distribution and show graphically how the prior
    has changed given the observation.}
  \end{itemize}
\end{homeworkProblem}


\clearpage


%--------------------------------------------------------------------------
%	PROBLEM 9
%--------------------------------------------------------------------------

\begin{homeworkProblem}
  Generate $N$ samples from a standard normal random variable and
  estimate the mean, variance, skewness, and kurtosis from the
  samples. Use $N=10,10^2,\cdot\cdot\cdot,10^4$, and discuss how
  the errors in the approximations behave as a function of N.\\

  Python script computing the likelihood of winning.
  \pythonscript{Problem9/Calculations}{Python Script for problem}

  \problemAnswer{
    Output of code:\\~\\
    You win 33\% of the time if you stay.\\
    You win 67\% of the time if you switch.
  }



  
\end{homeworkProblem}


\clearpage

%--------------------------------------------------------------------------
%	PROBLEM 10
%--------------------------------------------------------------------------

\begin{homeworkProblem}
  Consider the joint PDF
  \begin{equation*}
    f(x,y)=e^{-x/y},\ \ \ \ x\in[0,\infty)\ \ y\in[0,\sqrt{2}].
  \end{equation*}
  Compute and plot the marginal PDFs for X and Y. Additionally,
  compute the conditional probability distributions, and make
  plots for $f(y|X=\mu_x)$ and $f(x|y=\mu_y)$.
\end{homeworkProblem}


\clearpage


%--------------------------------------------------------------------------
This is an example citation \cite{Tatro2013}.
\bibliography{references} 
\bibliographystyle{plain} 

\end{document}
