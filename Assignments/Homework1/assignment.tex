%%%%%%%%%%%%%%%%%%%%%%%%%%%%%%%%%%%%%%%%%
% Programming/Coding Assignment
% LaTeX Template
%
% This template has been downloaded from:
% http://www.latextemplates.com
%
% Original author:
% Ted Pavlic (http://www.tedpavlic.com)
%
% Note:
% The \lipsum[#] commands throughout this template generate dummy text
% to fill the template out. These commands should all be removed when 
% writing assignment content.
%
% This template uses a Perl script as an example snippet of code, most other
% languages are also usable. Configure them in the "CODE INCLUSION 
% CONFIGURATION" section.
%
%%%%%%%%%%%%%%%%%%%%%%%%%%%%%%%%%%%%%%%%%

%----------------------------------------------------------------------------------------
%	PACKAGES AND OTHER DOCUMENT CONFIGURATIONS
%----------------------------------------------------------------------------------------

\documentclass{article}

\usepackage{fancyhdr} % Required for custom headers
\usepackage{lastpage} % Required to determine the last page for the footer
\usepackage{extramarks} % Required for headers and footers
\usepackage[usenames,dvipsnames]{color} % Required for custom colors
\usepackage{graphicx} % Required to insert images
\usepackage{listings} % Required for insertion of code
\usepackage{courier} % Required for the courier font
\usepackage{lipsum} % Used for inserting dummy 'Lorem ipsum' text into the template
\usepackage{hyperref}
\usepackage{amsmath}
% Margins
\topmargin=-0.45in
\evensidemargin=0in
\oddsidemargin=0in
\textwidth=6.5in
\textheight=9.0in
\headsep=0.25in

\linespread{1.1} % Line spacing

% Set up the header and footer
\pagestyle{fancy}
\lhead{\hmwkAuthorName} % Top left header
\chead{\hmwkClassShort\ (\hmwkClassInstructor)} % Top center head
%\rhead{\firstxmark} % Top right header
\rhead{\hmwkTitle}
\lfoot{\lastxmark} % Bottom left footer
\cfoot{} % Bottom center footer
\rfoot{Page\ \thepage\ of\ \protect\pageref{LastPage}} % Bottom right footer
\renewcommand\headrulewidth{0.4pt} % Size of the header rule
\renewcommand\footrulewidth{0.4pt} % Size of the footer rule

\setlength\parindent{0pt} % Removes all indentation from paragraphs

%----------------------------------------------------------------------------------------
%	CODE INCLUSION CONFIGURATION
%----------------------------------------------------------------------------------------

\definecolor{MyDarkGreen}{rgb}{0.0,0.4,0.0} % This is the color used for comments
\lstloadlanguages{Perl,Python} % Load Perl syntax for listings, for a list of other languages supported see: ftp://ftp.tex.ac.uk/tex-archive/macros/latex/contrib/listings/listings.pdf
\lstset{language=Perl, % Use Perl in this example
        frame=single, % Single frame around code
        basicstyle=\small\ttfamily, % Use small true type font
        keywordstyle=[1]\color{Blue}\bf, % Perl functions bold and blue
        keywordstyle=[2]\color{Purple}, % Perl function arguments purple
        keywordstyle=[3]\color{Blue}\underbar, % Custom functions underlined and blue
        identifierstyle=, % Nothing special about identifiers                                         
        commentstyle=\usefont{T1}{pcr}{m}{sl}\color{MyDarkGreen}\small, % Comments small dark green courier font
        stringstyle=\color{Purple}, % Strings are purple
        showstringspaces=false, % Don't put marks in string spaces
        tabsize=5, % 5 spaces per tab
        %
        % Put standard Perl functions not included in the default language here
        morekeywords={rand},
        %
        % Put Perl function parameters here
        morekeywords=[2]{on, off, interp},
        %
        % Put user defined functions here
        morekeywords=[3]{test},
       	%
        morecomment=[l][\color{Blue}]{...}, % Line continuation (...) like blue comment
        numbers=left, % Line numbers on left
        firstnumber=1, % Line numbers start with line 1
        numberstyle=\tiny\color{Blue}, % Line numbers are blue and small
        stepnumber=5 % Line numbers go in steps of 5
}


\lstset{language=Python, % Use Python in this example
        frame=single, % Single frame around code
        basicstyle=\small\ttfamily, % Use small true type font
        keywordstyle=[1]\color{Blue}\bf, % Python functions bold and blue
        keywordstyle=[2]\color{Purple}, % Python function arguments purple
        keywordstyle=[3]\color{Blue}\underbar, % Custom functions underlined and blue
        identifierstyle=, % Nothing special about identifiers                                         
        commentstyle=\usefont{T1}{pcr}{m}{sl}\color{MyDarkGreen}\small, % Comments small dark green courier font
        stringstyle=\color{Purple}, % Strings are purple
        showstringspaces=false, % Don't put marks in string spaces
        tabsize=3, % 5 spaces per tab
        %
        % Put standard Python functions not included in the default language here
        morekeywords={rand},
        %
        % Put Python function parameters here
        morekeywords=[2]{on, off, interp},
        %
        % Put user defined functions here
        morekeywords=[3]{test},
       	%
        morecomment=[l][\color{Blue}]{...}, % Line continuation (...) like blue comment
        numbers=left, % Line numbers on left
        firstnumber=1, % Line numbers start with line 1
        numberstyle=\tiny\color{Blue}, % Line numbers are blue and small
        stepnumber=5 % Line numbers go in steps of 5
}


% Creates a new command to include a perl script, the first parameter is the filename of the script (without .pl), the second parameter is the caption
\newcommand{\perlscript}[2]{
\begin{itemize}
\item[]\lstinputlisting[caption=#2,label=#1]{#1.pl}
\end{itemize}
}
\newcommand{\pythonscript}[2]{
\begin{itemize}
\item[]\lstinputlisting[caption=#2,label=#1]{#1}
\end{itemize}
}


%----------------------------------------------------------------------------------------
%	DOCUMENT STRUCTURE COMMANDS
%	Skip this unless you know what you're doing
%----------------------------------------------------------------------------------------

% Header and footer for when a page split occurs within a problem environment
\newcommand{\enterProblemHeader}[1]{
\nobreak\extramarks{#1}{#1 continued on next page\ldots}\nobreak
\nobreak\extramarks{#1 (continued)}{#1 continued on next page\ldots}\nobreak
}

% Header and footer for when a page split occurs between problem environments
\newcommand{\exitProblemHeader}[1]{
\nobreak\extramarks{#1 (continued)}{#1 continued on next page\ldots}\nobreak
\nobreak\extramarks{#1}{}\nobreak
}

\setcounter{secnumdepth}{0} % Removes default section numbers
\newcounter{homeworkProblemCounter} % Creates a counter to keep track of the number of problems

\newcommand{\homeworkProblemName}{}
\newenvironment{homeworkProblem}[1][Problem \arabic{homeworkProblemCounter}]{ % Makes a new environment called homeworkProblem which takes 1 argument (custom name) but the default is "Problem #"
\stepcounter{homeworkProblemCounter} % Increase counter for number of problems
\renewcommand{\homeworkProblemName}{#1} % Assign \homeworkProblemName the name of the problem
\section{\homeworkProblemName} % Make a section in the document with the custom problem count
\enterProblemHeader{\homeworkProblemName} % Header and footer within the environment
}{
\exitProblemHeader{\homeworkProblemName} % Header and footer after the environment
}

\newcommand{\problemAnswer}[1]{ % Defines the problem answer command with the content as the only argument
\noindent\framebox[\columnwidth][c]{\begin{minipage}{0.98\columnwidth}#1\end{minipage}} % Makes the box around the problem answer and puts the content inside
}

\newcommand{\homeworkSectionName}{}
\newenvironment{homeworkSection}[1]{ % New environment for sections within homework problems, takes 1 argument - the name of the section
\renewcommand{\homeworkSectionName}{#1} % Assign \homeworkSectionName to the name of the section from the environment argument
\subsection{\homeworkSectionName} % Make a subsection with the custom name of the subsection
\enterProblemHeader{\homeworkProblemName\ [\homeworkSectionName]} % Header and footer within the environment
}{
\enterProblemHeader{\homeworkProblemName} % Header and footer after the environment
}

%----------------------------------------------------------------------------------------
%	NAME AND CLASS SECTION
%----------------------------------------------------------------------------------------

\newcommand{\hmwkTitle}{Assignment\ 1} % Assignment title
\newcommand{\hmwkDueDate}{Tuesday,\ October\ 4,\ 2016} % Due date
\newcommand{\hmwkClass}{NUEN 647\\ Uncertainty Quantification for Nuclear Engineering} % Course/class
\newcommand{\hmwkClassTime}{Tu/Th 11:10am} % Class/lecture time
\newcommand{\hmwkClassInstructor}{Dr. McClarren} % Teacher/lecturer
\newcommand{\hmwkAuthorName}{Paul Mendoza} % Your name
\newcommand{\hmwkClassShort}{NUEN 647 UQ for Nuclear Engineering}
%----------------------------------------------------------------------------------------
%	TITLE PAGE
%----------------------------------------------------------------------------------------

\title{
\vspace{2in}
\textmd{\textbf{\hmwkClass\ \hmwkTitle}}\\
\normalsize\vspace{0.1in}\small{Due\ on\ \hmwkDueDate}\\
\vspace{0.1in}\large{\textit{\hmwkClassInstructor}}
\vspace{3in}
}

\author{\textbf{\hmwkAuthorName}}
\date{} % Insert date here if you want it to appear below your name

%----------------------------------------------------------------------------------------

\begin{document}

\maketitle

%----------------------------------------------------------------------------------------
%	TABLE OF CONTENTS
%----------------------------------------------------------------------------------------

%\setcounter{tocdepth}{1} % Uncomment this line if you don't want subsections listed in the ToC

\newpage
\tableofcontents
\newpage

Complete the exercises in the Chapter 2 notes. Be sure to include discussion of results
where appropriate. You may use any tools that are approrpriate to solving the problem.

%----------------------------------------------------------------------------------------
%	PROBLEM 1
%----------------------------------------------------------------------------------------

% To have just one problem per page, simply put a \clearpage after each problem

\begin{homeworkProblem}
  Show that the transformation in equation \ref{eq:1} results
  in a standard normal
  random variable by computing the mean and variance of z.

  \begin{equation} \label{eq:1}
    z = \frac{x-\mu}{\sigma}
  \end{equation}

  An important special case of the expectation value is the mean which
  is the expected value of $x$. It is often denoted as $\mu$,

  \begin{equation*}
    \mu=E[x]=\int_{-\infty}^\infty xf(x)dx
  \end{equation*}

  where $x$ is a realization of a random sample and $f(x)$ is
  the probability density function (PDF) for the random variable.
  For a normal distribution,

  \begin{equation*}
    f(x)=\frac{1}{\sigma\sqrt{2\pi}}e^{\frac{-(x-\mu)^2}{2\sigma^2}}
  \end{equation*}
  \problemAnswer{
  For the sake of the transformation, the value of z substitutes for
  $x$, the realization of a random sample (not the PDF because we are
  transforming that distribution). Therefore, the mean for
  z is:

  \begin{equation*}
    \mu_z=\int_{-\infty}^\infty \frac{x-\mu}{\sigma}
           \frac{1}{\sigma\sqrt{2\pi}}e^{\frac{-(x-\mu)^2}{2\sigma^2}}dx
  \end{equation*}

  If $u=(x-\mu)^2$ and $\frac{du}{2}=(x-\mu)dx$ (note that the limits
  change from $(-\infty,\infty)$ to $(\infty,\infty)$ - but that seems
  fishy to me so I will change it back after integration).

  \begin{equation*}
    \mu_z=\int_{\infty}^\infty
    \frac{1}{2\sigma^2\sqrt{2\pi}}e^{\frac{-u}{2\sigma^2}}du
    =\left|\frac{-1}{\sqrt{2\pi}}e^{\frac{-u}{2\sigma^2}}
      \right|_\infty^\infty
  \end{equation*}
  \begin{equation*}
    \mu_z
    =\left|\frac{-1}{\sqrt{2\pi}}e^{\frac{-(x-\mu)^2}{2\sigma^2}}
    \right|_{-\infty}^\infty=\frac{-1}{\sqrt{2\pi}}(e^{-\infty}-
    e^{-\infty}) =\boxed{0}
  \end{equation*}
  }
  \problemAnswer{
  The variance is defined as:

  \begin{equation*}
    Var(X)=E[(X-\mu)^2]
  \end{equation*}

  Substituting Eq. \ref{eq:1} for $X$, (but not for the pdf - I could
  be wrong about that)

  \begin{equation*}
    Var(X)=E[(\frac{x-\mu}{\sigma}-\mu)^2]=
    E\left[\left(\frac{x-\mu-\mu\sigma}{\sigma}\right)^2\right]=
    \frac{1}{\sigma^2}(E[x^2]-2\mu E[x]-2\mu\sigma E[x]+
    \mu^2E[1]+\mu^2\sigma^2+2\mu^2\sigma)
  \end{equation*}

  Noting that above it was proven that $E[x]=\mu$ and given that
  the definition of $E[1]=1$ and assuming that
  $E[x^2]=\sigma^2+\mu^2$ (will solve on next page)

  \begin{equation*}
    \frac{1}{\sigma^2}(\sigma^2+\mu^2-2\mu^2-2\mu^2\sigma+\mu^2
    +\mu^2\sigma^2+2\mu^2\sigma)=\frac{1}{\sigma^2}(\sigma^2+\mu^2\sigma^2)
    =\boxed{1+\mu^2=1}
  \end{equation*}
 
  This is assuming that $\mu=0$. Which was shown above.
  }
  \clearpage

  \begin{equation*}
    E[x^2]=\int_{-\infty}^{\infty} \frac{x^2}{\sigma\sqrt{2\pi}}
           e^{\frac{-(x-\mu)^2}{2\sigma^2}}
  \end{equation*}

  If $t=\frac{(x-\mu)}{\sqrt{2}\sigma}$ and $\sqrt{2}\sigma dt=dx$ and
  $x=t\sqrt{2}\sigma+\mu$
  then (limits of integration don't change)
  
  \begin{equation*}
    E[x^2]=\int_{-\infty}^{\infty} \frac{\left(t\sqrt{2}\sigma+\mu
      \right)^2}{\sqrt{\pi}}e^{-t^2}dt=
    \frac{1}{\sqrt{\pi}}\int_{-\infty}^{\infty}\left(2\sigma^
    2\left(t^2e^{-t^2}\right)+
    2\sqrt{2}\sigma\mu\left(te^{-t^2}\right)+\mu^2\left(e^{-t^2}\right)\right)
  \end{equation*}

  According to wolfram alpha

  \begin{equation*}
    \int_{-\infty}^\infty t^2e^{-t^2}=\frac{\sqrt{\pi}}{2}
  \end{equation*}

  \begin{equation*}
    \int_{-\infty}^\infty te^{-t^2}=0
  \end{equation*}

  \begin{equation*}
    \int_{-\infty}^\infty e^{-t^2}=\sqrt{\pi}
  \end{equation*}

  Which simplifies the above to $\sigma^2+\mu^2$.
%Listing \ref{Problem1/homework_example} shows a Perl script.

%\perlscript{Problem1/homework_example}{Sample Perl Script With Highlighting}

%\pythonscript{Problem1/Calculations}{Sample python script no .py}

%\pythonscript{Problem1/Functions.py}{Sample python script no .py}

\end{homeworkProblem}

\clearpage

%----------------------------------------------------------------------------------------
%	PROBLEM 2
%----------------------------------------------------------------------------------------

\begin{homeworkProblem}
%\lipsum[2]
  Consider the random variables $X\sim U(-1,1)$ and $Y\sim X^2$.
  Are these independent random variables? What is their covariance?
  \\~\\
  If two random variables, X and Y, are independent, they satisfy the
  following condition:
  \href{http://stattrek.com/random-variable/independence.aspx?Tutorial=AP}
       {$^{link}$}

  \begin{itemize}
    \item{$P(X|Y)=P(X)$, for all values of $X$ and $Y$.}
  \end{itemize}
  \problemAnswer{
  The PDF for X is:

  \begin{equation*}
    f_X(x)=\frac{1}{(1-(-1))}=0.5\ \ \ x \in [-1,1] 
  \end{equation*}

  The PDF for Y is:

  \begin{equation*}
    f_Y(x)=f_X(x)^2=0.25\ \ \ x \in [-1,1] 
  \end{equation*}

  I am confused, because at this point the PDF for Y does not integrate
  to one, but to 0.5...I probably did something wrong (anyway continue)
  \\~\\
  There is a special case for a collection of random variables where
  the joint PDF can be factored into the product of individual PDFs as:

  \begin{equation*}
    f(\vec{x})=f(x,y)={\displaystyle \prod_{i=1}^{p} f(x_i)}=f_X(x)*f_Y(x)=
    0.5*0.25=0.125\ \ \ x,y \in [-1,1]
  \end{equation*}
  
  We can define the probability distribution of Y provided X=x as

  \begin{equation*}
    f(y|X=x)=\frac{f(x,y)}{\int_{-\infty}^{\infty}f(x,y)dy}=
             \frac{0.125}{\int_{-1}^{1}0.125dy}=0.5
  \end{equation*}

  Are X and Y independant?

  \begin{align*}
    P(Y|X) &\stackrel{?}{=}P(Y)\\
     0.5 &\neq 0.25
  \end{align*}
  If what I did was correct...then X and Y shown to be dependant.
  }
  \\
  \problemAnswer{
  The measure for how random variables change together is called
  the covariance and the covariance between $X$ and $Y$ is written
  as $\sigma_{XY}$.
   
  \begin{align*}
    \sigma_{XY} &=E[(X-\mu_X)(Y-\mu_Y)]\\
                &=\int_{-1}^{1}dx\int_{-1}^{1}dy (x-\mu_x)(y-\mu_y)f(x,y)\\
                &=\int_{-1}^{1}dx\int_{-1}^{1}dy (xy-0.125x-0.5y+0.0625)
                 0.125\\
                &=0.125\int_{-1}^{1}dx\left| (\frac{xy^2}{2}-
                 0.125xy-0.25y^2+0.0625y)\right|_{-1}^1  
  \end{align*}
  }
%% \problemAnswer{
%% \begin{center}
%% \includegraphics[width=0.75\columnwidth]{Problem2/example_figure} % Example image
%% \end{center}

%% \lipsum[3-5]
%% }
\end{homeworkProblem}

\clearpage

%----------------------------------------------------------------------------------------



This is an example citation \cite{Tatro2013}.



\bibliography{references} % Use the NIHGrant.bib file for the reference list
\bibliographystyle{plain} 




\end{document}
