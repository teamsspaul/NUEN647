
\documentclass[11pt,notitlepage]{article}

\usepackage{amsfonts, amsmath, amsthm, amssymb} % For math fonts
\usepackage{booktabs} % Top and bottom rules for table
\usepackage{wrapfig} % Allows in-line images
\usepackage[labelfont=bf]{caption} % Make figure numbering in captions bold
\usepackage{graphicx} % Required for including images
\usepackage[top=0.7in,bottom=1in,left=1in,right=1in]{geometry}
\pagestyle{empty} % Remove page numbering
\newcommand{\tss}{\textsuperscript}
\newcommand{\tsbs}{\textsubscript}
\usepackage{float}
\hyphenation{ionto-pho-re-tic iso-tro-pic fortran} % Specifies custom hyphenation points for words or words that shouldn't be hyphenated at all
\newcommand\tab[1][0.6cm]{\hspace*{#1}}

\renewcommand{\thesection}{\Roman{section}}

\begin{document}

\vspace*{0.5cm}

\noindent{\fontsize{16}{19.2}\selectfont
  \textbf{NUEN 647 Final Project}}

\vspace{1cm}
Uncertainty quantification of depletion calculations
for specific isotopes using ORIGEN.

%------------------------------------------------------------
%	Introduction
%------------------------------------------------------------
 
\section{Introduction}

\tab
Determing composition of irradiated fuel is of importance
for a myriad of reasons. Whether for flux calculations,
reprocessing, or irradiation history verification,
calculating fuel composition requires a Bateman solver,
and a means for building a sparse matrix.

Applications using these compositions rarely report the
uncertainty associated with results, even when inputs,
such as flux shape, fission yield, cross sections,
and half-lives have varying degrees of uncertainty. A further
source of error in this calculation is due to the multi-group
approximation, but will not be explored here.

Several isotope concentrations, of interest to the writer,
will calculated as a function of burnup
with the depletion code ORIGEN for a thermal
and fast system using depleted uranium.
The uncertainty of these concentrations
will then be determined.

%------------------------------------------------------------
%	Objectives
%------------------------------------------------------------

\section{Objectives}

\begin{enumerate}
\item{Build ORIGEN model for fast and thermal system which
  calculates concentrations of isotopes shown in
  Table \ref{Table:1}.}
\item{Determine how to vary fission yields for calculation}
\item{Determine how to vary cross section and or flux spectrum
  inputs for calculation}
\item{Determine how to vary half-life information for calculation}
\item{Create a sampling space for all possible variations of
  calculations}
\item{Determine importance of various uncertain parameters
  by running the code a number of times randomly sampling
  the sample space (still not 100\% sure how to do this -
  not even 50\% sure how to do this)}
\end{enumerate}

\begin{table}[H]
  \begin{center}
    \caption{Isotope solve list.}
    \label{Table:1}
    \begin{tabular}{l l l}
      \toprule
      \tss{133}Cs & \tss{136}Ba & \tss{153}Eu\\
      \tss{134}Cs & \tss{138}Ba & \tss{154}Eu\\
      \tss{135}Cs & \tss{149}Sm & \tss{239}Pu\\
      \tss{137}Cs & \tss{150}Sm & \tss{242}Pu\\
      \tss{148}Nd & \tss{106}Rh & \tss{125}Sb\\
      \bottomrule
    \end{tabular}
  \end{center}
\end{table}

%------------------------------------------------------------
%	Quantities of Interest and uncertain parameters
%------------------------------------------------------------

\section{Quantities of Interest and Uncertain Parameters}

Quantities of interest are shown in Table \ref{Table:1} above.
Uncertain parameters are listed below:

\begin{itemize}
\item{Fission yield}
\item{Cross sections}
\item{Half-lives}
\end{itemize}

%------------------------------------------------------------
%	Prediction
%------------------------------------------------------------

\section{Prediction}

The first major prediction for this project is that half-lives
will not have a large impact on results because they are
relatively well known. Secondly, \tss{125}Sb is notorious
for being difficult to calculate, I would would predict that
there would be large uncertainties due to uncertainties in
the cross section data.

%----------------------------------------------------------------------------------------

\end{document}
